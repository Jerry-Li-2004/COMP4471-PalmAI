\section{Introduction}
\label{sec:intro}

Fortune-telling through palm reading, known as palmistry, has been practiced for centuries. 
However, such methods lack scientific support and are regarded by many as pseudoscience. 
There is also a lack of evidence for the connection between palm line features and the predictions stemming from this form of fortune-telling.

In the age of machine learning, large datasets can be leveraged along with machine learning techniques, such as deep neural networks, to uncover hidden characteristics and connections that were previously difficult to observe.
Under this predicate, we propose that there may be statistical connections between palm features and palmistry fortune-telling results.

In this project, we propose PalmReader.ai, an application to perform palmistry fortune-telling using an image of a palm. 
We construct a full pipeline to predict scores for aspects of the user’s life, such as character, emotional predispositions, health status, and life path.
Our pipeline is composed of the following components.

\begin{itemize}
    \item \textbf{Vision Transformer (ViT)}: A vision transformer serves to extract features to vector embeddings that encode geometric features of palm lines in the input image.
    \item \textbf{Scoring system}: We score different aspects of life by examining the geometric features of the detected palm lines, using a features extracted from the vision transformer.
    \item \textbf{LLM inference}: We produce, from the and scores, a report in natural language that outputs fortune-telling content.
\end{itemize}

We obtain satisfactory results from experiments with various architectures, with the final mean squared error (MSE) loss at around 0.028.
Inference using an LLM provides accurate reports on fortune-telling results based on our framework's predicted scores.


\section{Related Work}
\label{sec:relatedwork}

\subsection{Palmistry with Machine Learning}
~\cite{2509.02248} proposes palmistry inference using random forests and linear SVM models.
While the problem formulation is similar, we note that this paper classifies a palm line to different fixed categories, 
that eventually output to predefined descriptions.

Building upon this idea, we perform score prediction for each palm line and aspect of life,
and utilize an LLM for natural language inference.
This gives more informative and personalized outputs with our method.

\subsection{U-Net Context Fusion Module}
~\cite{2102.12127} proposes a deep learning-based solution to efficiently detect and segment palm lines from an image of a palm.
This paper proposes a novel component, the Context Fusion Module (CFM), which is integrated into the U-Net as a bottleneck.
The CFM captures both global and local contextual features from an image of the palm,
using an attention mechanism to capture global and local contextual features from the palm image.
Experimental results showed that the proposed method showed large advantages over traditional palm line segmentation methods.

Inspired by this method, we include a feature fusion module, combining features from regional feature extractors and global feature extractors.
This allows us to extract information from local to global levels of the image for further analysis and scoring.