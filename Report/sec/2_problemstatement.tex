\section{Problem Formulation}
\label{sec:problem}

\subsection{Four Primary Lines for Palm Line Analysis}
Palm line analysis is an ancient practice that interprets the lines on the palm of the hand to glean insights into an individual's personality traits, emotional tendencies, and life experiences.
The four primary lines—Heart, Head, Life, and Fate—are central to this practice, each offering unique perspectives into various aspects of human life.
All of our following tasks resolving these four lines~\cite{palmreading}:

\begin{itemize}
    \item \textbf{Heart Line}: Positioned at the top of the palm, the Heart Line is believed to reflect emotional stability, romantic perspectives, and cardiac health. 
    \item \textbf{Head Line}: Running horizontally across the palm, the Head Line represents intellectualism, learning style, and communication abilities. 
    \item \textbf{Life Line}: Encircling the thumb, the Life Line is traditionally associated with physical health, general well-being, and major life changes. 
    \item \textbf{Fate Line}: Often running vertically from the base of the palm, the Fate Line indicates the degree to which external circumstances influence an individual's life path. 
\end{itemize}

\begin{figure}[t]
  \centering
   \includegraphics[width=0.4\linewidth]{Images/palmlines.png}

   \caption{Different palm lines.}
   \label{fig:onecol}
\end{figure}

Therefore, one of our primary objectives is to successfully segment these primary palm lines. 
Achieving precise segmentation is essential before we can undertake further analytical tasks. 
This foundational step ensures that subsequent analysis will be based on accurate and distinct representations of each line.


\subsection{Feature Extraction and Text Decoder}
Following the segmentation of the palm lines, a critical subsequent step involves utilizing a text decoder to meticulously analyze the shapes and other significant characteristics that may influence the analysis. 
This detailed examination is essential before the data is passed to a large language model for comprehensive interpretation and final analysis.

% \begin{figure}[t]
%   \centering
%     \fbox{\rule{0pt}{2in} \rule{0.9\linewidth}{0pt}}
%    % \includegraphics[width=0.4\linewidth]{sec/palmlines.png}

%    \caption{An example of text output from the text decoder.}
%    \label{fig:onecol}
% \end{figure}


\subsection{Quantified Reasoning}
To enhance the reasoning process and mitigate any uncertainties, it is essential to implement a scoring system for the features identified in the palm lines. 
This scoring will assess various attributes, such as indicators of luck, health, and other relevant characteristics. 
By quantifying these features, we can provide a more structured and objective analysis that informs subsequent interpretations and enhances the overall reliability of the insights derived from the palm analysis.


\subsection{Final Analysis}
After obtaining scores for various attributes and elaborating on the text descriptions of the palm lines, the final and critical step is to generate a comprehensive, user-friendly analysis. 
This task will be accomplished using a large language model (LLM), which will synthesize the scores and textual insights into a coherent narrative. 
The LLM will provide an accessible interpretation of the analysis, enabling users to easily understand the implications of the palm features concerning aspects such as luck, health, and personal insights. 
By leveraging the capabilities of the LLM, we aim to deliver a polished and informative report that enhances the user experience and facilitates deeper engagement with the results of the palm analysis.

% \begin{figure}[t]
%   \centering
%     \fbox{\rule{0pt}{2in} \rule{0.9\linewidth}{0pt}}
%    % \includegraphics[width=0.4\linewidth]{sec/palmlines.png}

%    \caption{An example of the final analysis.}
%    \label{fig:onecol}
% \end{figure}